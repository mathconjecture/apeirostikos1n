\input{preamble.tex}
\input{definitions.tex}

\everymath{\displaystyle}
\pagestyle{vangelis}

\begin{document}

\begin{center}
  \minibox[c]{\large \bfseries \textcolor{Col1}{Ακολουθίες}\\ \large 
  \textcolor{Col1}{Ασκήσεις}}
\end{center}

\vspace{\baselineskip}


\setcounter{chapter}{1}
\section{Φραγμένες Ακολουθίες}

\begin{enumerate}
  \item Να δείξετε ότι η ακολουθία $ a_{n} = (-1)^{n}\frac{1}{2n} $ είναι 
    φραγμένη.
    \hfill Απ: $ \abs{a_{n}} \leq \frac{1}{2} $ 
  \item Να δείξετε ότι η ακολουθία $ a_{n} = \frac{5 \cos^{3}{n}}{n+2} $ 
    είναι φραγμένη.
    \hfill Απ: $ \abs{a_{n}} < \frac{5}{2}  $ 
  \item Να δείξετε ότι η ακολουθία $ a_{n} = \frac{\cos{n} + n \sin{n}}{n^{2}} $ 
    είναι φραγμένη. 
    \hfill Απ: $ \abs{a_{n}} \leq 2 $ 
  \item Να δείξετε ότι η ακολουθία $ a_{n} = \frac{n}{3^{n}} $ είναι 
    φραγμένη. 
    \hfill Απ: $ \abs{a_{n}}< \frac{1}{2} $
  \item Να δείξετε ότι η ακολουθία $ a_{n} = \frac{n!}{n^{n}} $ είναι 
    φραγμένη. 
    \hfill Απ: $ 0 \leq a_{n} \leq 1 $ 
  \item Να δείξετε ότι η ακολουθία $ a_{n} = 1 + \frac{1}{1!} +
    \frac{1}{2!} + \frac{1}{n!} $ είναι άνω φραγμένη.
    \hfill Απ: $ a_{n} < 3 $ 
    % \item Να δείξετε ότι η ακολουθία $ a_{n} = \frac{3 \sin{3n}}{n^{2}} $ 
    % είναι φραγμένη.
    % \hfill Απ: $ \abs{a_{n}} \leq 3 $ 
  \item Να δείξετε ότι η ακολουθία $ a_{1} = 3, \; a_{n+1} =
    \frac{a_{n}+4}{2}, \; \forall n \in \mathbb{N} $ είναι άνω φραγμένη.
    \hfill Απ: $ a_{n} < 4 $ 
  \item Να δείξετε ότι η ακολουθία $ a_{1} = \sqrt{2}, \; a_{n+1} =
    \sqrt{2+ a_{n}}, \; \forall n \in \mathbb{N} $ είναι φραγμένη.
    \hfill Απ: $ 0 < a_{n} < 2$ 
  \item Να δείξετε ότι η ακολουθία $ a_{n} = 2^{n} $ δεν είναι άνω 
    φραγμένη.
    % \item Να δείξετε ότι η ακολουθία $ a_{n} = \frac{n^{3} + \sin{5n}}{n} $ δεν είναι 
    %     φραγμένη.
  \item Να δείξετε ότι η ακολουθία $ a_{n} = \frac{n^{2}}{3n+ \sin^{2}{n}} $ δεν 
    είναι άνω φραγμένη.
\end{enumerate}

\section{Μονότονες Ακολουθίες}

\begin{enumerate}
  \item Να δείξετε ότι ακολουθία $ a_{n} = \frac{n}{5n-1} $ είναι 
    γνησίως φθίνουσα.
  \item Να δείξετε ότι ακολουθία $ a_{n} = \frac{n}{3^{n}} $ είναι 
    γνησίως φθίνουσα.
  \item Να δείξετε ότι ακολουθία $ a_{n} = \frac{2^{n}}{n!} $ είναι 
    γνησίως φθίνουσα.
  \item Να δείξετε ότι ακολουθία $ a_{n} = \frac{2n^{2}-1}{n} $ είναι γνησίως 
    αύξουσα.
  \item Να δείξετε ότι η ακολουθία $ a_{1}=0, \; a_{n+1}= 
    \frac{2 a_{n}+4}{3}, \; \forall n \in \mathbb{N} $ είναι γνησίως αύξουσα.
  \item Να δείξετε ότι ακολουθία $ a_{1}=1, \; a_{n} = \sqrt{a_{n}+1}, \; 
    \forall n \in \mathbb{N}$ είναι γνησίως αύξουσα.
    % \item Να δείξετε ότι ακολουθία $ a_{n} = \frac{1}{1\cdot 2} + \frac{1}{2\cdot 3} 
    % + \cdots + \frac{1}{n(n+1)} $ είναι γνησίως αύξουσα.
  \item Να δείξετε ότι ακολουθία $ a_{n} =  \frac{(-1)^{n}}{n^{2}+2} $ 
    δεν είναι μονότονη.
\end{enumerate}

\section{Ορισμός του Ορίου}

\begin{enumerate}
  \item Να δείξετε με τη βοήθεια του ορισμού τα παρακάτω όρια.
    \begin{enumerate}[i)]
      \item $ \lim_{n \to \infty} \frac{n}{3n-1} = \frac{1}{3} $
      \item $ \lim_{n \to \infty} \frac{n+2}{n^{2}} = 0 $
      \item $ \lim_{n \to \infty} \frac{3n -2}{2n+1} = \frac{3}{2} $ 
      \item $ \lim_{n \to \infty} \frac{5n-4}{2-3n} = - \frac{5}{3} $ 
      \item $ \lim_{n \to \infty} \frac{n^{2}+n}{n^{2}+3} = 1 $ 
      \item $ \lim_{n \to \infty} \frac{\sin{\frac{n^{3}}{3}}}{n^{3}} $
      \item $ \lim_{n \to \infty} \frac{\sin{n} + \cos{3n}}{n^{2}} = 0 $
      \item $ \lim_{n \to \infty} (\sqrt{n+2} - \sqrt{n}) = 0 $
    \end{enumerate}

  \item Να δείξετε ότι οι παρακάτω ακολουθίες δεν συγκλίνουν.
    \begin{enumerate}[i)]
      \item $ a_{n} = (-1)^{n} \frac{n}{n+1}  $
      \item $ a_{n} = (-1)^{n} \frac{n+3}{2n}  $
      \item $  a_{n} = \lambda n, \; \lambda >0$
    \end{enumerate}
\end{enumerate}

\section{Άλγεβρα και θεωρήματα Ορίων}

\begin{enumerate}

  \item Να υπολογιστούν τα παρακάτω όρια.
    \begin{enumerate}[i)]
      \item $ \lim_{n \to \infty} \frac{n^{2}+3n}{n^{2}+2n+1} $ \hfill Απ: 1 
      \item $ \lim_{n \to \infty} \sqrt[3]{\frac{n^{3}+n}{n^{3}+2n}} $ 
        \hfill Απ: 1 
    \end{enumerate}

  \item Να υπολογιστούν τα παρακάτω όρια με τη βοήθεια του Κριτηρίου 
    Παρεμβολής.

    \begin{enumerate}[i)]
      \item $ \lim_{n \to \infty} \frac{(-1)^{n}}{n^{2}+2n}  $ \hfill Απ: 0  
      \item $ \lim_{n \to \infty} (\sqrt{n+2} - \sqrt{n})  $ \hfill Απ:0
      \item $ \lim_{n \to \infty} (\sqrt{n^{3}+4} - \sqrt{n^{3}+1})  $ \hfill Απ:0
      \item $ \lim_{n \to \infty} \frac{4 \sin^{3}{n} + 3 \cos^{2}{n}}{n^{2}} $ 
        \hfill Απ:0
      \item $ \lim_{n \to \infty} \frac{\cos{n} + 3 \sin{4n}}{ 2
        \sqrt{n} -1} $ \hfill Απ: 0  
      \item $ \lim_{n \to \infty} \sqrt[n]{3^{n}+4^{n}+n} $ \hfill Απ: 4 
      \item $ \lim_{n \to \infty} \sqrt[n]{\frac{n^{2}}{3n^{2}+2}} $ \hfill Απ: 1 
      \item $ \lim_{n \to \infty} \sqrt[n]{n^{2}+n} $ \hfill Απ: 1 
    \end{enumerate}

  \item Να υπολογιστούν τα παρακάτω όρια με τη βοήθεια του ορίου 
    $ \lim_{n \to \infty} \left(1+ \frac{1}{n}\right)^{n}=e $

    \begin{enumerate}[i)]
      \item $ \lim_{n \to \infty} \left(1+ \frac{1}{n-2}\right)^{n}, 
        n \geq 3 $ 
        \hfill Απ: $e$  
      \item $ \lim_{n \to \infty} \left(1 + \frac{1}{3n}\right)^{n} $ 
        \hfill Απ: $ \sqrt[3]{e} $ 
      \item $ \lim_{n \to \infty} \left(1+ \frac{2}{n}\right)^{n} $ 
        \hfill Απ: $ e^{2} $ 
      \item $ \lim_{n \to \infty} \left(\frac{2n +3}{2n} \right)^{3n+2}  $
        \hfill Απ: $ e^{4}\cdot \sqrt{e} $ 
      \item $ \lim_{n \to \infty}\left(1-\frac{1}{n^{2}} \right)^{n} $ 
        \hfill Απ: 1 
      \item $ \lim_{n \to \infty} \left(\frac{n^{2}-1}{n^{2}+1} \right)^{n^{2}} $
        \hfill Απ: $ \frac{1}{e^{2}} $ 
    \end{enumerate}
\end{enumerate}




Να υπολογιστούν τα παρακάτω όρια.

\begin{enumerate}[i)]
  \item $ \lim_{n \to \infty} \sqrt[n]{3^{n}+5^{n}+7^{n}}  $ \hfill Απ: 7 
  \item $ \lim_{n \to \infty} \sqrt[n]{\left(\frac{3}{5} \right)^{n} + 
    \left(\frac{5}{6} \right)^{n}} $ \hfill Απ: $ \frac{5}{6} $ 
  \item $ \lim_{n \to \infty} \frac{n!}{n^{n}} $ \hfill Απ: 0  
  \item $ \lim_{n \to \infty} \frac{6\cdot 3^{n}-7 \cdot 4^{n}+8}
    {5^{n}+3\cdot 2^{n}+1} $ \hfill Απ: 0  
  \item $ \lim_{n \to \infty} \frac{4^{n+3}}{\sqrt{4^{4n-2}}} $ \hfill Απ: 0  
  \item $ \lim_{n \to \infty} \frac{3n}{3^{n}(n^{2}+2)} $ \hfill Απ: 0  
  \item (\bfseries Θέμα:2018) $ \lim_{n \to \infty} \sqrt[n]{\frac{1}{2^{n}}+ 
    \frac{1}{3^{n}}} $ \hfill Απ: $ \frac{1}{2} $ 
  \item (\bfseries Θέμα:2018) $ \lim_{n \to \infty} \frac{n^{3}}{3^{n}} $ 
    \hfill Απ: $ 0 $
  \item (\bfseries Θέμα:2019) $ \lim_{n \to \infty} \sqrt[n]{\frac{1}{3^{n}} + 
    \frac{1}{4^{n}} + \frac{1}{5^{n}}} $ \hfill Απ: $ \frac{1}{3} $
  \item (\bfseries Θέμα:2019) $ \lim_{n \to \infty} \frac{n^{4}+5n-6}{2^{n}} $ 
    \hfill Απ: 0 
\end{enumerate}

\end{document}
