\input{preamble_ask.tex}
\input{definitions_ask.tex}
\input{tikz.tex}

\pagestyle{askhseis}
\everymath{\displaystyle}

\begin{document}

\begin{center}
  \minibox[c]{\large \bfseries \textcolor{Col1}{Ακολουθίες} \\ \large
  \textcolor{Col1}{Ασκήσεις}}
\end{center}

\vspace{\baselineskip}


\setcounter{chapter}{1}

\section*{Σύγκλιση Ακολουθίας}

\begin{enumerate}
  \item Να δείξετε ότι οι παρακάτω ακολουθίες δεν συγκλίνουν.
    \begin{enumerate}[i)]
      \item $ a_{n} = (-1)^{n} \frac{n}{n+1} $ 
      \item $ a_{n} = (-1)^{n} \frac{n+3}{2n} $ 
      \item $ a_{n} = \lambda n, \; \lambda > 0 $ 
    \end{enumerate}
    \begin{proof}
    \item {}
      \begin{enumerate}[i)]
        \item Θεωρούμε τις υπακολουθίες $(a_{2n})_{n \in \mathbb{N}} $ 
          και $(a_{2n-1})_{n \in \mathbb{N}} $. Έχουμε,

          \begin{align*} 
            \lim_{n \to \infty} a_{2n} = \lim_{n \to \infty}  (-1)^{2n} 
            \frac{2n}{2n+1} = \lim_{n \to \infty}  \frac{2n}{2n+1} = 
            \lim_{n \to \infty} \frac{2n}{n(2 + \frac{1}{n})} = 
            \lim_{n \to \infty} \frac{2}{2 + \frac{1}{n}} = 
            \frac{2}{2+0} = 1 
          \end{align*}

          \begin{align*} 
            \lim_{n \to \infty} a_{2n-1} = \lim_{n \to \infty} 
            (-1)^{2n-1} \frac{2n-1}{2n-1 + 1} = \lim_{n \to \infty} - 
            \frac{2n-1}{2n} = -\lim_{n \to \infty} 
            \frac{n (2 - \frac{1}{n})}{2n} = - \lim_{n \to \infty} 
            \frac{2 - \frac{1}{n} }{2} = -1 
          \end{align*}

          Επειδή $ \lim_{n \to \infty} a_{2n} \neq \lim_{n \to \infty} a_{2n-1} $
          η ακολουθία $ (a_{n})_{n \in \mathbb{N}} $ δε συγκλίνει.

        \item Θεωρούμε τις υπακολουθίες $(a_{2n})_{n \in \mathbb{N}} $ 
          και $(a_{2n-1})_{n \in \mathbb{N}} $. Έχουμε,

          \begin{align*}
            \lim_{n \to \infty} a_{2n} = (-1)^{2n} \lim_{n \to \infty} 
            \frac{2n+3}{2\cdot 2n} = \lim_{n \to \infty} \frac{2n+3}{4n} = 
            \lim_{n \to \infty} \frac{n(2+ \frac{3}{n})}{4n} = 
            \lim_{n \to \infty} \frac{2 + \frac{3}{n}}{4} = \frac{1}{2} 
          \end{align*}

          \begin{align*}
            \lim_{n \to \infty} a_{2n-1} = \lim_{n \to \infty} (-1)^{2n-1} 
            \frac{2n-1 +3}{2 \cdot (2n-1)} = \lim_{n \to \infty} - 
            \frac{2n+2}{4n-2} = - \lim_{n \to \infty} 
            \frac{n(2+ \frac{2}{n})}{n(4- \frac{2}{n})} = - \lim_{n \to \infty}
            \frac{2 + \frac{2}{n}}{4 - \frac{2}{n}} = - \frac{1}{2}
          \end{align*}

          Επειδή $ \lim_{n \to \infty} a_{2n} \neq \lim_{n \to \infty} a_{2n-1} $
          η ακολουθία $ (a_{n})_{n \in \mathbb{N}} $ δε συγκλίνει.

        \item Η ακολουθία $(\lambda n)_{n \in \mathbb{N}} $ δε συγκλίνει, διότι 
          δεν είναι φραγμένη. Πράγματι, γιατί αν $ (a_{n})_{n \in \mathbb{N}} $
          είναι φραγμένη, τότε
          \[
            \abs{a_{n}} = \abs{\lambda n} \overset{\lambda >0}{=} 
            \lambda n \leq a, \; \forall n \in \mathbb{N} \Leftrightarrow 
            n \leq \frac{a}{\lambda}, \; \forall n \in \mathbb{N} \; 
            \text{άτοπο, γιατι $ \mathbb{N} $ όχι άνω φραγμένο}
          \] 
      \end{enumerate}    
    \end{proof}
\end{enumerate}


\section{Άλγεβρα και Θεωρήματα Ορίων}


\begin{enumerate}
  \item Να υπολογιστούν τα παρακάτω όρια.

    \begin{enumerate}[i)]
      \item $ \lim_{n \to \infty} \frac{n^{2}+3n}{n^{2}+2n+1} = 
        \lim_{n \to \infty} \frac{n^{2}(1+ \frac{3}{n})}{n^{2}(1+ \frac{2}{n} 
        + \frac{1}{n^{2}})} = \lim_{n \to \infty} 
        \frac{1+ \frac{3}{n}}{1+ \frac{2}{n} + \frac{1}{n^{2}}} = 
        \frac{1+0}{1+0+0} = 1 $ 

      \item $ \lim_{n \to \infty} \sqrt[3]{\frac{n^{3}+n}{n^{3}+2n}} = 
        \sqrt[3]{\lim_{n \to \infty} \frac{n^{3}(1+ 
        \frac{1}{n^{2}})}{n^{3}(1+ \frac{2}{n^{2}})}} = 
        \sqrt[3]{\lim_{n \to \infty} \frac{1+ \frac{1}{n^{2}}}{1+
        \frac{2}{n^{2}}}} = \sqrt[3]{\frac{1+0}{1+0}} = \sqrt[3]{1} = 1  $
    \end{enumerate}

  \item Να υπολογιστούν τα παρακάτω όρια με τη βοήθεια του Κριτηρίου Παρεμβολής.

    \begin{enumerate}[i)]
      \item $ \lim_{n \to \infty} \frac{(-1)^{n}}{n^{2}+2n} $
        \begin{proof}
        \item {}
          $ \abs{a_{n}} = \abs{\frac{(-1)^{n}}{n^{2}+2n}} =
          \frac{\abs{(-1)^{n}}}{\abs{n^{2}+2n}} = \frac{1}{n^{2}+2n} \leq 
          \frac{1}{n^{2}} $

          Όμως $ \lim_{n \to \infty} \frac{1}{n^{2}} = 0 $, άρα και $ 
          \lim_{n \to \infty} a_{n} = \lim_{n \to \infty} 
          \frac{(-1)^{n}}{n^{2}+2n} = 0$
        \end{proof}

      \item $ \lim_{n \to \infty} (\sqrt{n+2} - \sqrt{n}) $
        \begin{proof}
        \item {}
          \begin{align*} 
            \abs{a_{n}} &= \abs{\sqrt{n+2} - \sqrt{n}} = \sqrt{n+2} - 
            \sqrt{n} =  \frac{(\sqrt{n+2} - \sqrt{n})(\sqrt{n+2} + 
            \sqrt{n})}{\sqrt{n+2} + \sqrt{n}} = 
            \frac{n+2-n}{\sqrt{n+2} + \sqrt{n}} \\
                        &= \frac{2}{\sqrt{n+2} + 
                        \sqrt{n} } \leq \frac{2}{\sqrt{n}}  
          \end{align*}

          Όμως $ \lim_{n \to \infty} \frac{2}{\sqrt{n}} = 0 $, άρα και 
          $ \lim_{n \to \infty} \sqrt{n+2} - \sqrt{n} = 0 $
        \end{proof}

      \item $ \lim_{n \to \infty} \left(\sqrt{n^{3}+4} - \sqrt{n^{3}+1}\right) $
        \begin{proof}
        \item {}
          \begin{align*}
            \abs{a_{n}} &= \abs{\sqrt{n^{3}+4} - \sqrt{n^{3}+1}} = 
            \sqrt{n^{3}+4} - \sqrt{n^{3}+1} = \frac{(\sqrt{n^{3}+4}-
              \sqrt{n^{3}+1})(\sqrt{n^{3}+4}+ 
            \sqrt{n^{3}+1})}{\sqrt{n^{3}+4} + \sqrt{n^{3}+1}} \\
                        &= \frac{n^{3}+4 - n^{3}-1}{\sqrt{n^{3}+4} + 
                          \sqrt{n^{3}+1}} = \frac{3}{\sqrt{n^{3}+4} + 
                        \sqrt{n^{3}+1}} <
                        \frac{3}{\sqrt{n^{3}+1}+ \sqrt{n^{3}+1}} = 
                        \frac{3}{2 \sqrt{n^{3}+1}} < 
                        \frac{3}{\sqrt{n^{3}}} \\
                        &= \frac{3}{n \sqrt{n}} 
          \end{align*}

          Όμως $ \lim_{n \to \infty} \frac{3}{n \sqrt{n}} = 
          \lim_{n \to \infty} \left(\frac{3}{n} \cdot 
          \frac{1}{\sqrt{n}}\right) = 
          0 \cdot 0 = 0 $, άρα και $ \lim_{n \to \infty} 
          \left(\sqrt{n^{3}+4} - \sqrt{n^{3}+1}\right) = 0 $ 
        \end{proof}

      \item $ \lim_{n \to \infty} \frac{4 \sin^{3}{n} + 3 \cos^{2}{n}}{n^{2}} $
        \begin{proof}
        \item {}
          \begin{align*}
            \abs{\frac{4 \sin^{3}{n} + 3 \cos^{2}{n}}{n^{2}}} 
                        &= \frac{\abs{4 \sin^{3}{n} + 3 \cos^{2}{n}}}{n^{2}} \leq 
                        \frac{\abs{4 \sin^{3}{n}} + \abs{3 \cos^{2}{n}}}{n^{2}} = 
                        \frac{4 \abs{\sin^{3}{n}} + 3 \abs{\cos^{2}{n}}}{n^{2}} \\ 
                        &= \frac{4 \abs{\sin{n}}^{3} + 3 \abs{\cos{n}}^{3}}{n^{2}} \leq 
                        \frac{4 \cdot 1^{3}+ 3 \cdot 1 ^{3}}{n^{2}} = \frac{7}{n^{2}} 
                        < \frac{7}{n} 
          \end{align*}

          Όμως $ \lim_{n \to \infty} \frac{7}{n} = 0 $, άρα και $ 
          \lim_{n \to \infty} \frac{4 \sin^{3}{n} + 3 \cos^{2}{n}}{n^{2}} = 0$
        \end{proof}

      \item $ \lim_{n \to \infty} \frac{\cos{n} + 3 \sin{4n}}{2 \sqrt{n}-1} $
        \begin{proof}
        \item {}
          \begin{align*}
            \abs{a_{n}} &= \abs{\frac{\cos{n} + 3 \sin{4n}}{2 \sqrt{n} -1}} 
            = \frac{\abs{\cos{n} + 3 \sin{4n}}}{2 \sqrt{n} -1} \leq 
            \frac{\abs{\cos{n}} + \abs{3 \sin{4n}}}{2 \sqrt{n} -1} = 
            \frac{\abs{\cos{n}} + 3 \abs{\sin{4n}}}{2 \sqrt{n} -1} \leq 
            \frac{1 + 3 \cdot 1}{2 \sqrt{n} -1} \\ 
                        &= \frac{4}{2 \sqrt{n} -1} 
          \end{align*}

          Όμως $ \lim_{n \to \infty} \frac{4}{\sqrt{n} (2 - 
          \frac{1}{\sqrt{n}})} = \lim_{n \to \infty}  
          \left(\frac{1}{\sqrt{n}} \cdot \frac{4}{(2- 
          \frac{1}{\sqrt{n}})}\right) = 0 \cdot \frac{4}{2-0} = 0 $
        \end{proof}

      \item $ \lim_{n \to \infty} \sqrt[n]{3^{n}+4^{n}+n} $
        \begin{proof}
        \item {}
          Ισχύει
          \begin{align*}
            4^{n} &\leq 3^{n}+4^{n}+n \leq 4^{n}+4^{n}+4^{n}, \; 
            \forall n \in \mathbb{N} \\
            4^{n} &\leq 3^{n}+4^{n}+n \leq 3\cdot 4^{n}, \; \forall n 
            \in \mathbb{N} \\
            \sqrt[n]{4^{n}} & \leq \sqrt[n]{3^{n}+4^{n}+n} \leq 
            \sqrt[n]{3\cdot 4^{n}}, \; 
            \forall n \in \mathbb{N} \\
            4 & \leq \sqrt[n]{3^{n}+4^{n}+n} \leq 4 \cdot \sqrt[n]{3}, \; 
            \forall n \in \mathbb{N}
          \end{align*}

          Όμως $ \lim_{n \to \infty} 4 = 4 $ και $ \lim_{n \to \infty} 4 
          \sqrt[n]{3} = 4 \cdot
          1 = 4$, άρα από Κριτήριο παρεμβολής και $ \lim_{n \to \infty}
          \sqrt[n]{3^{n}+4^{n}+n} = 4 $
        \end{proof}

      \item $ \lim_{n \to \infty} \sqrt[n]{\frac{n^{2}}{3n^{2}+2}} $
        \begin{proof}
        \item {}
          Ισχύει
          \begin{align*}
            \frac{1}{5} = \frac{n^{2}}{5n^{2}} \leq 
            \frac{n^{2}}{3n^{2}+2n^{2}} \leq \frac{n^{2}}{3n^{2}+2} 
            \leq \frac{n^{2}}{3n^{2}} = \frac{1}{3}, 
            \; \forall n \in \mathbb{N}
          \end{align*}
          Επομένως 
          \[
            \sqrt[n]{\frac{1}{5}} \leq \sqrt[n]{\frac{n^{2}}{3n^{2}+2}} 
            \leq \sqrt[n]{\frac{1}{3}}, \; \forall n \in \mathbb{N}
          \] 
          Όμως $ \lim_{n \to \infty} \sqrt[n]{\frac{1}{5}} = 
          \lim_{n \to \infty}
          \sqrt{\frac{1}{3}} = 1 $, άρα από Κριτήριο Παρεμβολής 
          και $ \lim_{n \to \infty} \sqrt[n]{\frac{n^{2}}{3n^{2}+2}} = 1 $
        \end{proof}

      \item $ \lim_{n \to \infty} \sqrt[n]{n^{2}+n} $
        \begin{proof}
        \item {}
          Ισχύει
          \[
            1 < n^{2}+n \leq n^{2}+n^{2} = 2n^{2}, \; \forall n \in \mathbb{N} 
          \] 
          Επομένως
          \[
            \sqrt[n]{1} < \sqrt[n]{n^{2}+n} \leq \sqrt[n]{2n^{2}}
          \]

          Όμως $ \lim_{n \to \infty} \sqrt[n]{1} = 1$ και $ \lim_{n \to \infty} 
          \sqrt[n]{2n^{2}} = \lim_{n \to \infty} (\sqrt[n]{2} \cdot \sqrt[n]{n^{2}}) =
          \lim_{n \to \infty} (\sqrt[n]{2} \cdot \sqrt[n]{n} \cdot \sqrt[n]{n}) = 
          1 \cdot 1 \cdot 1 = 1$, άρα και $ \lim_{n \to \infty} \sqrt[n]{n^{2}+n} = 1 $
        \end{proof}
    \end{enumerate}

  \item Να υπολογιστούν τα παρακάτω όρια.

    \begin{enumerate}[i)]
      \item $ \lim_{n \to \infty} \sqrt[n]{3^{n} + 5^{n}+7^{n}} $ 
        \begin{proof}
          \begin{gather*}
            7^{n} < 3^{n}+5^{n}+7^{n} < 7^{n}+7^{n}+7^{n} = 3\cdot 7^{n} 
            \Leftrightarrow \\
            \sqrt[n]{7^{n}} < \sqrt[n]{3^{n}+5^{n}+7^{n}} < 
            \sqrt[n]{3 \cdot 7^{n}} \Leftrightarrow \\
            7 < \sqrt[n]{3^{n}+5^{n}+7^{n}} < 7\cdot \sqrt[n]{3} 
            \Leftrightarrow \\
          \end{gather*}
          Όμως $ \lim_{n \to \infty} 7 = 7 $ και $ \lim_{n \to \infty} 7 
          \cdot \sqrt[n]{3} = 7 \cdot 1 = 7$, άρα και $ 
          \lim_{n \to \infty} \sqrt[n]{3^{n}+5^{n}+7^{n}} = 7$
        \end{proof}

      \item $ \lim_{n \to \infty} \sqrt[n]{\left(\frac{3}{5}\right)^{n}+
        \left(\frac{5}{6}\right)^{n}} $
        \begin{proof}
          Ισχύει ότι $ \frac{3}{5} < \frac{5}{6} $, οπότε
          \begin{gather*}
            \left(\frac{5}{6}\right)^{n} < \left(\frac{3}{5}\right)^{n}+ 
            \left(\frac{5}{6}\right)^{n} < \left(\frac{5}{6}\right)^{n} + 
            \left(\frac{5}{6}\right)^{n} = 2 \cdot 
            \left(\frac{5}{6}\right)^{n} 
            \Leftrightarrow \\
            \sqrt[n]{\left(\frac{5}{6}\right)^{n}} < 
            \sqrt[n]{\left(\frac{3}{5}\right)^{n}+ 
            \left(\frac{5}{6}\right)^{n}} < 
            \sqrt[n]{2\cdot \left(\frac{5}{6}\right)^{n}} 
            \Leftrightarrow \\
            \frac{5}{6} < \sqrt[n]{\left(\frac{3}{5}\right)^{n}+ 
            \left(\frac{5}{6}\right)^{n}} < \frac{5}{6} \cdot 
            \sqrt[n]{2} 
          \end{gather*}
          Όμως $ \lim_{n \to \infty} \frac{5}{6} = \frac{5}{6} $ και 
          $ \lim_{n \to \infty} \left(\frac{5}{6} \cdot \sqrt[n]{2}\right) = 
          \frac{5}{6} \cdot 1 = \frac{5}{6} $, άρα και $ 
          \lim_{n \to \infty} \sqrt[n]{\left(\frac{3}{5} \right)^{n} + 
          \left(\frac{5}{6}\right)^{n}} = \frac{5}{6} $
        \end{proof}

      \item $ \lim_{n \to \infty} \frac{6\cdot 3^{n}-7 \cdot 4^{n}+8}
        {5^{n}+3\cdot 2^{n}+1} $ 
        \begin{proof}
          \begin{align*}
            a_{n} 
                        &= \frac{6\cdot 3^{n}-7 \cdot 4^{n}+8} {5^{n}+3
                        \cdot 2^{n}+1} = 
                        \frac{5^{n}\left(6\cdot \frac{3^{n}}{5^{n}}  - 7 \cdot 
                            \frac{4^{n}}{5^{n}} + 
                            \frac{8}{5^{n}}\right)}{5^{n}\left(1 + 3 
                        \cdot \frac{2^{n}}{5^{n}} + \frac{1}{5^{n}}\right)} = 
                        \frac{6\cdot (\frac{3}{5} )^{n}-7\cdot (\frac{4}{5} )^{n} +8
                          \cdot ( \frac{1}{5} )^{n}}{1 + 3\cdot (\frac{2}{5} )^{n}+ 
                        (\frac{1}{5} )^{n}} \\
                        & \quad \xrightarrow{n \to \infty}  
                        \frac{6 \cdot 0 -7 \cdot 0 + 8 \cdot 0}{1 + 3 \cdot 0 + 0} = 0  
          \end{align*} 
        \end{proof}

      \item $ \lim_{n \to \infty} \frac{4^{n+3}}{\sqrt{4^{4n-2}}} $ 
        \begin{proof}
          \[
            \abs{a_{n}} = \abs{\frac{4^{n+3}}{\sqrt{4^{4n-2}}}} = 
            \frac{4^{n}\cdot 
            4^{3}}{\rlap{\cancel{\phantom{--}}}{\sqrt{4^{\cancel{2}(2n-1)}}}} = 
            \frac{4^{n} \cdot 4^{3}}{4^{2n-1}} = 
            \frac{4^{n}\cdot 4^{3}}{4^{2n}\cdot 4^{-1}} = 
            \frac{4^{3}}{4^{n}\cdot 
            \frac{1}{4}} = \frac{4^{4}}{4^{n}} = 4^{4}\cdot 
            \left(\frac{1}{4} \right) ^{n}
          \] 
          Όμως $ \lim_{n \to \infty} \left(\frac{1}{4}\right)^{n} = 0  $, 
          δηλαδή $ \lim_{n \to \infty} \abs{a_{n}} = \lim_{n \to \infty}
          \abs{\frac{4^{n+3}}{\sqrt{4^{4n-2}}}} = 0 $. 

          Άρα από γνωστή πρόταση και
          $ \lim_{n \to \infty} a_{n} =  \lim_{n \to \infty} 
          \frac{4^{n+3}}{\sqrt{4^{4n-2}}} = 0 $ 
        \end{proof}
    \end{enumerate}
\end{enumerate}



\end{document}
