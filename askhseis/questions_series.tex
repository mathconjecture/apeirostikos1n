\input{preamble_ask.tex}
\input{definitions_ask.tex}
\input{tikz}

\pagestyle{askhseis}


\begin{document}

\begin{center}
  \minibox{\large\bfseries \textcolor{Col2}{Σειρές (Ερωτήσεις)}}
\end{center}

% \section*{Ακολουθίες}

\vspace{\baselineskip}

\hfill \textcolor{Col1}{\textbf{Σωστό}} \quad \textcolor{Col1}{\textbf{Λάθος}}
\begin{enumerate}[itemsep=.5\baselineskip]
  \item \textcolor{Col1}{Αν μια σειρά είναι συγκλίνουσα, τότε η ακολουθία των μερ.
    αθροισμάτων της είναι φραγμένη}. 
    \hfill $\textcolor{Col1}{\blacksquare} \qquad \qquad \textcolor{Col1}{\square}$

    Απ: $ \sum_{n=1}^{\infty} a_{n} \; \text{συγκλίνει} \; \Leftrightarrow \lim_{n \to
    \infty} S_{n} = S \in \mathbb{R} $, άρα αφού είναι συγκλίνουσα, είναι και φραγμένη. 

  \item \textcolor{Col1}{Αν η ακολουθία των μερ. αθροισμάτων μιας σειράς, δεν είναι
    φραγμένη, τότε η σειρά αποκλίνει}.
    \hfill $\textcolor{Col1}{\blacksquare} \qquad \qquad \textcolor{Col1}{\square}$

    Απ: είναι το αντιθετοαντίστροφο της παραπάνω πρότασης.

  \item \textcolor{Col1}{Αν $ \sum_{n=1}^{\infty} a_{n} $ συγκλίνει, τότε 
      $ {(a_{n})}_{n \in \mathbb{N}} $ είναι φραγμένη}.
    \hfill $\textcolor{Col1}{\blacksquare} \qquad \qquad \textcolor{Col1}{\square}$

    Απ: $ \sum_{n=1}^{\infty} a_{n} $ συγκλίνει $ \Rightarrow \lim_{n \to \infty} a_{n} =
    0 $, δηλ. $ {(a_{n})}_{n \in \mathbb{N}} $ συγκλίνει, άρα φραγμένη

  \item \textcolor{Col1}{Αν η $ {(a_{n})}_{n \in \mathbb{N}} $ είναι μηδενική, τότε η
    σειρά $ \sum_{n=1}^{\infty} a_{n} $ συγκλίνει}.
    \hfill $\textcolor{Col1}{\square} \qquad \qquad \textcolor{Col1}{\blacksquare}$

    Απ: π.χ. $ \lim_{n \to \infty} \frac{1}{n} = 0 $ όμως $ \sum_{n=1}^{\infty}
    \frac{1}{n} $ αποκλίνει 

  \item \textcolor{Col1}{Αν $ a_{n} \geq 0, \; \forall n \in
    \mathbb{N} $, τότε η σειρά $ \sum_{n=1}^{\infty} a_{n} $ απειρίζεται θετικά}.
    \hfill $\textcolor{Col1}{\square} \qquad \qquad \textcolor{Col1}{\blacksquare}$

    Απ: θεώρημα: αν $ a_{n} \geq 0, \; \forall n \in \mathbb{N} $, τότε η σειρά 
    $ \sum_{n=1}^{\infty} a_{n} $ \textbf{συγκλίνει} ή απειρίζεται θετικά.

  \item \textcolor{Col1}{Αν $ a_{n} > 0, \; \forall n \in \mathbb{N} $ και $ {(S_{n})}_{n
    \in \mathbb{N}}$ φραγμένη, τότε η σειρά $ \sum_{n=1}^{\infty} a_{n} $ συγκλίνει}.
    \hfill $\textcolor{Col1}{\blacksquare} \qquad \qquad \textcolor{Col1}{\square}$

    Απ: είναι το παραπάνω θεώρημα, (ενώ αν \textbf{δεν} είναι φραγμένη, απειρίζεται 
    θετικά). 

  \item \textcolor{Col1}{Οι $ \sum_{n=1}^{\infty} a_{n} $ και $
      \sum_{n=1}^{\infty} a_{n+ n_{0}}, \; n_{0} \in \mathbb{N} $ 
    παρουσιάζουν ίδια συμπεριφορά ως προς τη σύγκλιση}.
    \hfill $\textcolor{Col1}{\blacksquare} \qquad \qquad \textcolor{Col1}{\square}$

    Απ: πρόταση: $ \sum_{n=1}^{\infty} a_{n} $ συγκλίνει $ \Leftrightarrow
    \sum_{n=n_{0}}^{\infty} a_{n} $ συγκλίνει (είναι αν και μόνον αν πρόταση).

  \item \textcolor{Col1}{Αν $ a_{n} < 0, \; \forall n \in \mathbb{N} $, τότε $
    \sum_{n=1}^{\infty} a_{n} = - \infty $}.
    \hfill $\textcolor{Col1}{\square} \qquad \qquad \textcolor{Col1}{\blacksquare}$

    Απ: αν $ {(S_{n})}_{n \in \mathbb{N}} $ είναι φραγμένη, θα συγκλίνει, 
    αλλιώς, θα απειρίζεται αρνητικά 

  \item \textcolor{Col1}{$ \sum_{n=1}^{\infty} a_{n} = a $ και $ \sum_{n=1}^{\infty}
      b_{n} = b, \Rightarrow \sum_{n=1}^{\infty} (\kappa a_{n} + \lambda b_{n}) 
    = k a + \lambda b $}.
    \hfill $\textcolor{Col1}{\blacksquare} \qquad \qquad \textcolor{Col1}{\square}$

    Απ: είναι πρόταση. 

  \item \textcolor{Col1}{αν $ \sum_{n=1}^{\infty} a_{n} $, $ \sum_{n=1}^{\infty} b_{n} $ 
    σειρές, με $ a_{n}=b_{n}, \; \forall n \geq n_{0} $, τότε ίδια συμπεριφορά ως προς τη
  σύγκλιση}.
    \hfill $\textcolor{Col1}{\blacksquare} \qquad \qquad \textcolor{Col1}{\square}$

    Απ: η σύγκλιση της σειράς δεν επηρεάζεται από την προσθήκη πεπερασμένου πλήθους
    αρχικών όρων.

  \item \textcolor{Col1}{Αν $ \sum_{n=1}^{\infty} a_{n}$ συγκλίνει και $
    \sum_{n=1}^{\infty} b_{n} $ αποκλίνει, τότε η 
  $ \sum_{n=1}^{\infty}(a_{n}+b_{n}) $ αποκλίνει}.
    \hfill $\textcolor{Col1}{\blacksquare} \qquad \qquad \textcolor{Col1}{\square}$

    Απ: είναι πρόταση 

  \item \textcolor{Col1}{Αν $ \sum_{n=1}^{\infty} a_{n}$ συγκλίνει, τότε 
    $ \sum_{n=1}^{\infty} \frac{1}{a_{n}} $ αποκλίνει.}
    \hfill $\textcolor{Col1}{\blacksquare} \qquad \qquad \textcolor{Col1}{\square}$

    Απ: $ \sum_{n=1}^{\infty} a_{n} $ συγκλ. $ \Rightarrow \lim_{n \to \infty} a_{n} = 0
    \Rightarrow \lim_{n \to \infty} \frac{1}{a_{n}} = \infty \neq 0 \Rightarrow
    \sum_{n=1}^{\infty} \frac{1}{a_{n}} $ αποκλίνει.

  \item \textcolor{Col1}{Αν $ \sum_{n=1}^{\infty} a_{n} $ συγκλίνει τότε $
    \sum_{n=1}^{\infty} \abs{a_{n}} $ συγκλίνει απολύτως}.
    \hfill $\textcolor{Col1}{\square} \qquad \qquad \textcolor{Col1}{\blacksquare}$

    Απ: θεώρημα: $ \sum_{n=1}^{\infty} \abs{a_{n}} $ συγκλίνει $ \Rightarrow
    \sum_{n=1}^{\infty} a_{n} $ συγκλίνει απολύτως.

  \item \textcolor{Col1}{Αν μια σειρά έχει θετικούς όρους, τότε \textbf{σύγκλιση} 
    και \textbf{απόλυτη σύγκλιση} είναι το ίδιο.}
    \hfill $\textcolor{Col1}{\blacksquare} \qquad \qquad \textcolor{Col1}{\square}$

    Απ: προφανώς, γιατί $ \sum_{n=1}^{\infty} a_{n} = \sum_{n=1}^{\infty} \abs{a_{n}} $ 

  \item \textcolor{Col1}{η $ \sum_{n=1}^{\infty} a^{n} $ αποκλίνει $ \Leftrightarrow
    \abs{a} \geq 1 $}.
    \hfill $\textcolor{Col1}{\blacksquare} \qquad \qquad \textcolor{Col1}{\square}$

    Απ: η γεωμετρική σειρά $ \sum_{n=1}^{\infty} a^{n} $ συγκλίνει 
    $ \Leftrightarrow \abs{a} < 1 $ και αποκλίνει αν $ \abs{a} \geq 1 $

  \item \textcolor{Col1}{η $ \sum_{n=1}^{\infty} \frac{1}{\sqrt[3]{n}} $ συγκλίνει}.
    \hfill $\textcolor{Col1}{\square} \qquad \qquad \textcolor{Col1}{\blacksquare}$

    Απ: είναι \textbf{γενικ. αρμονική} σειρά, $ \sum_{n=1}^{\infty} 
    \frac{1}{\sqrt[3]{n}} = \sum_{n=1}^{\infty} \frac{1}{n^{1/3}} $ 
    αποκλίνει, γιατί $ p=1/3 < 1 $.

  \item \textcolor{Col1}{Το άθροισμα της σειράς $ \sum_{n=0}^{\infty}
    \frac{1}{(1+ \mathrm{e})^{n}} = \mathrm{e} $}.
    \hfill $\textcolor{Col1}{\square} \qquad \qquad \textcolor{Col1}{\blacksquare}$

    Απ: είναι \textbf{γεωμετρική} με $ \lambda = \frac{1}{1+ \mathrm{e}} < 1 
    $, άρα  $ \sum_{n=1}^{\infty} \frac{1}{(1+ \mathrm{e})^{n}} = \sum_{n=1}^{\infty}
    \left(\frac{1}{1+ \mathrm{e}}\right)^{n} = \frac{1+ \mathrm{e}}{\mathrm{e}} $

    

\end{enumerate}
\end{document}
