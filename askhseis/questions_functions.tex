\input{preamble_ask.tex}
\input{definitions_ask.tex}
\input{tikz}

\pagestyle{askhseis}


\begin{document}

\begin{center}
  \minibox{\large\bfseries \textcolor{Col2}{Συναρτήσεις (Ερωτήσεις)}}
\end{center}

% \section*{Ακολουθίες}

\vspace{\baselineskip}

\hfill \textcolor{Col1}{\textbf{Σωστό}} \quad \textcolor{Col1}{\textbf{Λάθος}}
\begin{enumerate}[itemsep=.5\baselineskip]
  \item \textcolor{Col1}{Αν $f$ είναι 1-1, τότε $ f' $ είναι 1-1}. 
    \hfill $\textcolor{Col1}{\square} \qquad \qquad \textcolor{Col1}{\blacksquare}$

    Απ: $ f(x)=x^{3} $ είναι 1-1. Όμως η $ f'(x) = 3x^{2} $ δεν είναι 1-1

  \item \textcolor{Col1}{Η συνάρτηση $ f(x)=x^{3}, \; \forall x \in [-1,0] $ 
    είναι περιττή}.
    \hfill $\textcolor{Col1}{\square} \qquad \qquad \textcolor{Col1}{\blacksquare}$

    Απ: $ \frac{1}{2} \not\in [-1,0] $. Θυμάμαι ότι $f$ περιττή 
    $ \Leftrightarrow \forall x \in A, -x \in A $ και $ f(-x)=f(x), \; \forall x \in A $ 

  \item \textcolor{Col1}{Αν $f$ είναι περιττή, τότε η $ f' $ είναι άρτια}.
    \hfill $\textcolor{Col1}{\blacksquare} \qquad \qquad \textcolor{Col1}{\square}$

    Απ: $ f(-x)=-f(x) \Rightarrow f'(-x)\cdot (-1) = - f'(x) \Rightarrow f'(-x) = f'(x)
    $, άρα άρτια.

  \item \textcolor{Col1}{Αν το σημείο $ (1,2) \in C_{f} $ και $f$ αντιστρέψιμη, 
    τότε το $ (2,1) \in C_{f^{-1}} $}.
    \hfill $\textcolor{Col1}{\blacksquare} \qquad \qquad \textcolor{Col1}{\square}$

    Απ: Η $f$ και η αντίστροφή της είναι συμμετρικές ως προς την ευθεία $ y=x $.

  \item \textcolor{Col1}{Αν $ f(x)=x+1, \; \forall x \in \mathbb{R} $ και 
      $ g(x)= \ln{(x-2)}, \; \forall x \in (1, + \infty) $, τότε το 
    $ D_{g \mathsmaller\circ f} = (2, + \infty) $}.
    \hfill $\textcolor{Col1}{\square} \qquad \qquad \textcolor{Col1}{\blacksquare}$

    Απ: $ D_{g \mathsmaller\circ f} = \{ x \in A \; : \; x+1 \in (2, \infty) \} = 
    \{ x \in A \; : \; x + 1 > 2 \} = \{ x \in A \; : \; x > 1 \} = (1, \infty) $

  \item \textcolor{Col1}{Κάθε γνησίως αύξουσα συνάρτηση είναι "1-1"}.
    \hfill $\textcolor{Col1}{\blacksquare} \qquad \qquad \textcolor{Col1}{\square}$

    Απ: Είναι πρόταση

  \item \textcolor{Col1}{Αν μιά συνάρτηση είναι περιοδική, τότε η περίοδος της είναι
    μοναδική}.
    \hfill $\textcolor{Col1}{\square} \qquad \qquad \textcolor{Col1}{\blacksquare}$

    Απ: Το $ 2p $ είναι επίσης περίοδος.

  \item \textcolor{Col1}{Αν $f$ είναι περιοδική, τότε και $ f' $ είναι περιοδική}.
    \hfill $\textcolor{Col1}{\blacksquare} \qquad \qquad \textcolor{Col1}{\square}$

    Απ: $ f(x+p)=f(x) \Rightarrow f'(x+p)\cdot 1 = f'(x) \Rightarrow f'(x+p)=f'(x) $

  \item \textcolor{Col1}{Η συνάρτηση Dirichlet είναι περιοδική}.
    \hfill $\textcolor{Col1}{\blacksquare} \qquad \qquad \textcolor{Col1}{\square}$

    Απ: $ p \in \mathbb{Q} \Rightarrow x \in \mathbb{Q} \Rightarrow f(x+p) = 1 = f(x), 
    \; x \not \in \mathbb{Q} \Rightarrow f(x+p)=0=f(x) $

  \item \textcolor{Col1}{$ \lim\limits_{x \to x_{0}} \abs{f(x)} = \abs{l} \Rightarrow
    \lim\limits_{x \to x_{0}} f(x) = l $}.
    \hfill $\textcolor{Col1}{\square} \qquad \qquad \textcolor{Col1}{\blacksquare}$

    Απ: π.χ. $ \lim\limits_{x \to \infty} \abs{(-1)^{n}} = 1 $, όμως 
    $  \cancel{\exists} \lim\limits_{x \to \infty} (-1)^{n} $ \; 
    (ισχύει, μόνο αν $ l=0 $).

  \item \textcolor{Col1}{Αν $f$ συνεχής στο $A$, τότε η $ \abs{f} $ με $ \abs{f}(x) 
    = \abs{f(x)}, \; \forall x \in A $ είναι συνεχής στο $A$}.
    \hfill $\textcolor{Col1}{\blacksquare} \qquad \qquad \textcolor{Col1}{\square}$

    Απ: είναι πρόταση. 

  \item \textcolor{Col1}{$ f \colon [0,1] \cup \{ 2 \} \to \mathbb{R} $ είναι 
    συνεχής στο 2}.
    \hfill $\textcolor{Col1}{\blacksquare} \qquad \qquad \textcolor{Col1}{\square}$

    Απ: Κάθε συνάρτηση, είναι συνεχής σε όλα τα \textbf{μεμονωμένα} σημεία του πεδίου 
    ορισμού της.

  \item \textcolor{Col1}{Αν $ a_{n} \in A $ με $ \lim\limits_{n \to \infty} a_{n}=a $,  
      αλλά $ \lim\limits_{n \to \infty} f(a_{n}) \neq f(a) $ τότε $ f \colon A \to 
    \mathbb{R}$ όχι συνεχής στο $ a \in A $}.
    \hfill $\textcolor{Col1}{\blacksquare} \qquad \qquad \textcolor{Col1}{\square}$

    Απ: είναι το αντιθετοαντίστροφο της Αρχής Μεταφοράς.

  \item \textcolor{Col1}{Αν $f$ είναι ομοιόμορφα συνεχής στο $A$, τότε $f$ είναι  
    συνεχής στο $A$}.
    \hfill $\textcolor{Col1}{\blacksquare} \qquad \qquad \textcolor{Col1}{\square}$

    Απ: είναι πρόταση. Το αντίστροφο δεν ισχύει.

  \item \textcolor{Col1}{Αν $f$ είναι συνεχής σε φραγμένο διάστημα, τότε η $f$ 
    είναι ομοιόμορφα συνεχής σε αυτό}.
    \hfill $\textcolor{Col1}{\square} \qquad \qquad \textcolor{Col1}{\blacksquare}$

    Απ: πρέπει να είναι \textbf{κλειστό} και φραγμένο διάστημα για να ισχύει η πρόταση.

  \item \textcolor{Col1}{Η $ f(x)=x^{2} $ είναι ομοιόμορφα συνεχής στο $ (2,4) $}.
    \hfill $\textcolor{Col1}{\blacksquare} \qquad \qquad \textcolor{Col1}{\square}$

    Απ: $ \exists \delta = \frac{\varepsilon}{8} > 0 \; : \; $ αν $ x \in (2,4) $ με 
    $ \abs{x- x_{0}} < \delta \Rightarrow \abs{x^{2}- x_{0}^{2}} = \abs{x- x_{0}}
    \abs{x+ x_{0}} < \frac{\varepsilon}{8} (4+4) = \varepsilon $

  \item \textcolor{Col1}{Η $ f(x)= \ln{x} $ είναι ομοιόμορφα συνεχής στο [1,2]}.
    \hfill $\textcolor{Col1}{\blacksquare} \qquad \qquad \textcolor{Col1}{\square}$

    Απ: είναι συνεχής σε κλειστό και φραγμένο διάστημα, άρα ομοιόμορφα συνεχής.

  \item \textcolor{Col1}{Αν μια συνάρτηση είναι συνεχής στο $ x_{0} $ τότε είναι και 
    παραγωγίσιμη στο $ x_{0} $}.
    \hfill $\textcolor{Col1}{\square} \qquad \qquad \textcolor{Col1}{\blacksquare}$

    Απ: Αν $f$ είναι παραγωγίσιμη στο $ x_{0} $, τότε είναι και συνεχής στο $ x_{0} $.

  \item \textcolor{Col1}{Αν $ f'(x_{0}) = 0 $, τότε $ f(x_{0}) $ είναι ακρότατη τιμή
    της $f$}.
    \hfill $\textcolor{Col1}{\square} \qquad \qquad \textcolor{Col1}{\blacksquare}$

    Απ: $ f(x)=x^{3} \Rightarrow f'(0)=0 $ όμως $ f(0) $ όχι ακρότατο της $f$. 

  \item \textcolor{Col1}{Αν $ f(x_{0}) $ ακρότατο της $f$, τότε η $f$ είναι
    παραγωγίσιμη στο $ x_{0} $}.
    \hfill $\textcolor{Col1}{\square} \qquad \qquad \textcolor{Col1}{\blacksquare}$

    Απ: Η $ f(x) = \abs{x} $ έχει ακρότατο για $ x=0 $, όμως $ f'(0) $, δεν υπάρχει.

  \item \textcolor{Col1}{Αν $f$ συνεχής και γν. αύξουσα στο $ [a,b] $, τότε $ f'(x)>0, \;
    \forall x \in [a,b] $}.
    \hfill $\textcolor{Col1}{\square} \qquad \qquad \textcolor{Col1}{\blacksquare}$

    Απ: Η $ f(x) = x^{3} $ συνεχής και γν. αύξουσα στο $ [-2,2] $, όμως $ f'(0)=0 $

  \item \textcolor{Col1}{Αν $ f''(x_{0}) = 0 $, τότε $ f(x_{0}) $ είναι σημείο καμπής 
    της $f$}.
    \hfill $\textcolor{Col1}{\square} \qquad \qquad \textcolor{Col1}{\blacksquare}$

    Απ: Για την $ f(x) = x^{4} $ ισχύει, $ f''(0)=0 $, όμως $ f(0) $ είναι τοπικό
    ελάχιστο.

  \item \textcolor{Col1}{Αν $ f \colon [1,3] \to (2,6) $ είναι επί, τότε η $f$ 
    είναι συνεχής}.
    \hfill $\textcolor{Col1}{\square} \qquad \qquad \textcolor{Col1}{\blacksquare}$

    Απ: $ f $ επί $ \Rightarrow f([1,3]) = (2,6) $. Άρα $f$ όχι συνεχής, 
    γιατί συνεχής εικόνα κλειστού διαστ. είναι κλειστό διάστ.

  \item \textcolor{Col1}{Αν $ f,g \colon A \to \mathbb{R} $ ομοιόμορφα συνεχείς τότε 
    $ f+g $ είναι ομοιόμορφα συνεχείς}.
    \hfill $\textcolor{Col1}{\blacksquare} \qquad \qquad \textcolor{Col1}{\square}$

    Απ: Είναι πρόταση.

  \item \textcolor{Col1}{Αν $ f,g \colon A \to \mathbb{R} $ ομοιόμορφα συνεχείς τότε 
    $ f\cdot g $ είναι ομοιόμορφα συνεχείς}.
    \hfill $\textcolor{Col1}{\square} \qquad \qquad \textcolor{Col1}{\blacksquare}$

    Απ: $ f=g=x $ ομοιόμορφα συνεχείς στο $\mathbb{R}$, όμως $ f\cdot g = x^{2} $ 
    δεν είναι ομοιόμορφα συνεχής στο $\mathbb{R}$.

\end{enumerate}
\end{document}

