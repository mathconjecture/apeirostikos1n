\documentclass[a4paper,table]{report}
\input{preamble.tex}
\input{definitions.tex}
\input{tikz.tex}
\input{myboxes.tex}

\newcommand{\twocolumnsidesss}[2]{\begin{minipage}[t]{0.49\linewidth}\raggedright
        #1
        \end{minipage}\hfill\begin{minipage}[t]{0.49\linewidth}\raggedright
        #2
    \end{minipage}
}


\pagestyle{askhseis}

\zexternaldocument*{chapter1}

\begin{document}

\refstepcounter{chapter}
\chapter*{Ακολουθίες}

\section{Ορισμός Ακολουθίας}

\begin{mybox1}
  \begin{dfn}
    \textcolor{Col1}{Ακολουθία} πραγματικών αριθμών ονομάζεται 
    κάθε συνάρτηση με πεδίο ορισμού τους φυσικούς αριθμούς. 
    \begin{align*}
      a \colon &\mathbb{N} \to \mathbb{R} \\
               &n \mapsto a(n)=a_{n} \quad (\text{ν-οστός όρος})
    \end{align*} 
    Οι ακολουθίες συμβολίζονται ως $ (a_{n})_{n \in \mathbb{N}} $ 
    ή $ ( a_{n} ) _{n=1}^{+\infty}$  ή $ \{ a_{n} \} _{n \in \mathbb{N}} $ ή 
    $ \{ a_{n} \} _{n=1}^{+\infty}$ , κλπ.
  \end{dfn}
\end{mybox1}

\begin{mybox1}
  \begin{dfn}
    \textcolor{Col1}{Σύνολο Τιμών} (Σ.Τ.) της ακολουθίας 
    $ (a_{n})_{n \in \mathbb{N}} $, ονομάζουμε το \textbf{σύνολο των όρων της}, 
    δηλαδή το σύνολο $ \{ a_{1}, a_{2}, \ldots, a_{n}, \ldots \} = \{ a_{n} \; : \; n
    \in \mathbb{N} \} $ το οποίο μπορεί να είναι πεπερασμένο ή άπειρο.
  \end{dfn}
\end{mybox1}

\begin{examples}
\item {}
  \begin{enumerate}[i)]
    \item Η ακολουθία $ Η ακολουθία a_{n} = n, \; \forall n \in \mathbb{N} $. 
      Έχει Σ.Τ.\  το σύνολο $  \{ 1,2,3, \ldots \} $.
    \item Η ακολουθία $ a_{n}=\left(\frac{1}{n}\right)_{n \in \mathbb{N}} $. 
      Έχει Σ.Τ.\ το σύνολο $  \left\{ 1, \frac{1}{2}, \frac{1}{3}, \ldots \right\} $.
    \item Η ακολουθία $ a_{n}= \{(-1)^{n}\}_{n=1}^{+ \infty}, $. Έχει Σ.Τ.\ 
      το σύνολο $ \{ -1,1 \} $.
    \item Η ακολουθία $ a_{n} = c, \; \forall n \in \mathbb{N}, c \in \mathbb{R} $.
      Έχει Σ.Τ.\ το σύνολο $ \{ c \} $ και ονομάζεται 
      \textcolor{Col1}{σταθερή ακολουθία}.
    \item Η ακολουθία $ a_{n}=2n, \; \forall n \in \mathbb{N} $. Έχει Σ.Τ.\ το 
      σύνολο $ \{ 2,4,6, \ldots, 2n, \ldots \} $, δηλαδή όλους τους
      \textcolor{Col1}{άρτιους} φυσικούς αριθμούς.
    \item Η ακολουθία $ a_{n}= 2n-1, \; \forall n \in \mathbb{N} $. Έχει Σ.Τ.\ το 
      σύνολο $ \{ 1,3,5, \ldots, 2n-1, \ldots \} $, δηλαδή όλους τους περιττούς
      φυσικούς αριθμούς.
    \item \label{ex:anadr} Η ακολουθία $ a_{1}= a_{2} = 1 $ και $ a_{n+2}=a_{n+1}
      +a_{n}, \; \forall n \in \mathbb{N}$. Έχει Σ.Τ.\ το σύνολο 
      $ \{ 1,1,2,3,5,8, 13,21,34, \ldots\} $.  Πρόκειται για την 
      \textcolor{Col1}{ακολουθία Fibonacci}. 
  \end{enumerate}
\end{examples}

\begin{rem}
\item {}
  \begin{enumerate}[i)]
    \item Ουσιαστικά οι ακολουθίες είναι άπειρες \textbf{λίστες} πραγματικών αριθμών.
    \item Η ακολουθία~\ref{ex:anadr}, όπου κάθε επόμενος όρος, ορίζεται με 
      τη βοήθεια του προηγούμενου, λέγεται
      \textcolor{Col1}{αναδρομική ακολουθία}.  Προτάσεις που αφορούν 
      αναδρομικές ακολουθίες, αποδεικνύονται με \textbf{Μαθηματική Επαγωγή}.
  \end{enumerate}
\end{rem}

\begin{mybox1}
  \begin{dfn}
    Δυο ακολουθίες, $(a_{n})_{n \in \mathbb{N}}$  και $ (b_{n})_{n \in \mathbb{N}} $ 
    ονομάζονται \textcolor{Col1}{ίσες}, αν 
    $ a_{n} = b_{n}, \; \forall n \in \mathbb{N} $.
  \end{dfn}
\end{mybox1}

\begin{mybox1}
  \begin{dfn}
    Οι πράξεις μεταξύ ακολουθιών, ορίζονται όπως ακριβώς και για τις συναρτήσεις.
  \end{dfn}
\end{mybox1}
\begin{example}
  Αν $ {(a_{n})}_{n \in \mathbb{N}} $ και $ {(b_{n})}_{n \in \mathbb{N}} $ ακολουθίες,
  τότε ορίζουμε το άθροισμα $ (a+b)_{n} = a_{n} + b_{n}, \; \forall n \in \mathbb{N} $
\end{example}


\section{Φραγμένες Ακολουθίες}

\begin{mybox1}
  \begin{dfn}
    Μια ακολουθία $ (a_{n})_{n \in \mathbb{N}} $ ονομάζεται:
    \begin{enumerate}[i)]
      \item \textcolor{Col1}{άνω φραγμένη} 
        $ \overset{\text{ορ.}}{\Leftrightarrow} \exists M \in 
        \mathbb{R} \; : \; a_{n} \leq M, \; \forall n \in \mathbb{N}$.
      \item \textcolor{Col1}{κάτω φραγμένη} 
        $ \overset{\text{ορ.}}{\Leftrightarrow} \exists m \in 
        \mathbb{R} \; : \; m \leq a_{n}, \; \forall n \in \mathbb{N}  $
      \item \textcolor{Col1}{φραγμένη} 
        $ \overset{\text{ορ.}}{\Leftrightarrow} $ είναι άνω και κάτω φραγμένη.
    \end{enumerate}
  \end{dfn}
\end{mybox1}

\begin{mybox3}
  \begin{prop}\label{prop:apolfragm}
    {$ (a_{n})_{n \in \mathbb{N}} $ φραγμένη $ \Leftrightarrow \exists M \in 
      \mathbb{R}, \; M>0 \; : \; \abs{a_{n}} \leq M, \; \forall n \in \mathbb{N} $ 
    (\textbf{απολύτως φραγμένη}).}
  \end{prop}
\end{mybox3}
\begin{proof}
\item {}
  \begin{description}
    \item [($ \Rightarrow $)] Έστω $ (a_{n})_{n \in \mathbb{N}} $ φραγμένη. Τότε η 
      $ (a_{n})_{n \in \mathbb{N}} $ είναι άνω και κάτω φραγμένη, δηλαδή υπάρχουν 
      $ m,M \in \mathbb{R} $ ώστε $ m \leq a_{n} \leq M, \; \forall n \in \mathbb{N} $.
      Θεωρούμε $ \mu = \max \{ \abs{m} , \abs{M} \} $. Τότε, η παραπάνω διπλή 
      ανισότητα, λαμβάνοντας υπόψιν και τις γνωστές ιδιότητες της απόλυτης τιμής, 
      γίνεται
      \[
        - \mu \leq - \abs{m} \leq m \leq a_{n} \leq M \leq \abs{M} \leq \mu , 
        \quad \forall n \in \mathbb{N}
      \]
      Επομένως $ \exists \mu > 0 $ ώστε 
      $ \abs{a_{n}} \leq \mu, \; \forall n \in \mathbb{N} $.

    \item[(Β' Τρόπος:)]
      Θεωρούμε $ \mu > \max \{ \abs{m} , \abs{M} \} $. Τότε, η παραπάνω διπλή 
      ανισότητα γίνεται,
      \[
        - \mu < m \leq a_{n} \leq M < \mu  
      \] 

    \item [($ \Leftarrow$)] Προφανώς, αν $( a_{n})_{n \in \mathbb{N}} $ είναι 
      απολύτως φραγμένη, τότε $ \exists M>0 $ ώστε $ -M \leq a_{n} \leq M, \; \forall n
      \in \mathbb{N} $, και άρα η ακολουθία είναι φραγμένη. 
  \end{description}
\end{proof}

\begin{rem}
  Πολλές φορές στη βιβλιογραφία, η πρόταση~\ref{prop:apolfragm}, 
  δίνεται και ως ορισμός της φραγμένης ακολουθίας.
\end{rem}

\begin{examples}
\item {}  
  \begin{enumerate}[i)]
    \item Η ακολουθία $ a_{n}= \frac{1}{n}, \; \forall n \in \mathbb{N} $, είναι 
      φραγμένη.
      Πράγματι, προφανώς, ισχύει ότι $ 0 \leq \frac{1}{n} \leq 1, \; \forall n \in 
      \mathbb{N} $. 

      Επίσης, $ \abs{\frac{1}{n}} = \frac{1}{n} \leq 1, \; \forall n \in \mathbb{N} $, 
      άρα είναι και απολύτως φραγμένη.

    \item Η ακολουθία $ a_{n}=(-1)^{n} \frac{2}{n}, \; \forall n \in \mathbb{N} $ 
      είναι απολύτως φραγμένη. Πράγματι,
      \[
        \abs{a_{n}} = \abs{(-1)^{n} \frac{2}{n}} = \abs{(-1)^{n}} 
        \cdot \abs{\frac{2}{n}} = \abs{-1}^{n} \cdot \frac{2}{n}
        = 1 \cdot \frac{2}{n} = \frac{2}{n} \leq 2, \; \forall n \in \mathbb{N}
      \] 

    \item Η ακολουθία $ a_{n}= \frac{3\sin{n} + \cos^{2}n}{n^{2}} , \; 
      \forall n \in \mathbb{N} $ είναι φραγμένη. Πράγματι, 

      \[
        \abs{a_{n}} = \abs{\frac{3\sin{n} + \cos^{2}{n}}{n^{2}}} = \frac{\abs{3
          \sin{n} + \cos^{2}{n}}}{n^{2}} \leq \frac{3\abs{\sin{n}} +
        \abs{\cos{n}}^{2}}{n^{2}} \leq \frac{3\cdot 1 + 1^{2}}{n^{2}} = \frac{4}{n^{2}}
        \leq 4
      \] 

    \item Η ακολουθία $ a_{n}= \frac{n}{2^{n}}, \; \forall n \in \mathbb{N} $ είναι 
      φραγμένη. Πράγματι, προφανώς, $ 0 \leq a_{n}, \; \forall n \in \mathbb{N} $. 
      \[
        2^{n} = (1+1)^{n} \geq 1+n > n, \; \forall n \in \mathbb{N} 
      \]
      Άρα 
      \[
        a_{n}= \frac{n}{2^{n}} < \frac{n}{n} = 1, \; \forall n \in \mathbb{N}
      \]

    \item Η ακολουθία $ a_{n}= \frac{(n-1)!}{n^{n}}, \; \forall n \in \mathbb{N} $
      είναι φραγμένη. Πράγματι, 

      $ a_{n} > 0, \; \forall n \in 
      \mathbb{N}$, άρα 0 κ.φ.\ της $( a_{n})_{n \in 
      \mathbb{N}} $. 
      Επίσης 
      \[
        a_{n}= \frac{(n-1)!}{n^{n}} = \frac{1 \cdot 2 
          \cdots (n-1)}{n^{n}} < \frac{\smash{\overbrace{n 
              \cdot n \cdots n} ^{n-1 \; 
        \text{φορές}}}}{n^{n}} = \frac{n^{n-1}}{n^{n}} =
        \frac{1}{n} \leq 1, \; \forall n \in \mathbb{N},
      \]
      άρα το 1 είναι α.φ.\ της $(a_{n})_{n \in \mathbb{N}}$. 

    \item Η ακολουθία $ a_{n}= 1 + \left(- \frac{1}{2} \right) + \left(- 
        \frac{1}{2}\right)^{2} + \cdots + \left(-\frac{1}{2} 
      \right) ^{n}, 
      \; \forall n \in \mathbb{N} $ είναι απολύτως φραγμένη. Πράγματι,

      Πρόκειται για το άθροισμα των $ n $ πρώτων όρων \textbf{γεωμετρικής προόδου} με 
      λόγο $ -\frac{1}{2} $. Έτσι
      \[ a_{n} = 1 + \left(- \frac{1}{2}\right) + \left(- \frac{1}{2} 
        \right)^{2} + \cdots + \left(- \frac{1}{2} \right)^{n} = 
        \frac{(- \frac{1}{2} )^{n+1}-1}{- \frac{1}{2}-1} = 
      \frac{2}{3} \left[1 - \left(- \frac{1}{2} \right)^{n}\right] \]
      Επομένως
      \[
        \abs{a_{n}} = \abs{\frac{2}{3} \left[1-\left(- \frac{1}{2} \right)^{n}
            \right]} = \frac{2}{3} \abs{1 - \left(- 
        \frac{1}{2}\right)^{n}} \leq 
        \frac{2}{3} \left(1 + \abs{-\frac{1}{2} }^{n} \right) = 
        \frac{2}{3} \left(1+ \frac{1}{2^{n}}\right) < \frac{2}{3}
        (1+1) = \frac{4}{3} 
      \] 

    \item Η ακολουθία $ a_{n}= 2n+5, \; \forall n \in \mathbb{N} $ είναι κάτω 
      φραγμένη.
      Πράγματι, $ 7 \leq 2n+5, \; \forall n \in \mathbb{N} $, άρα το 
      7 είναι κ.φ.\ της $ (a_{n} )_{n \in \mathbb{N}} $.

      Προσοχή, η ακολουθία $ a_{n}= 2n+5, \; \forall n \in \mathbb{N} $ δεν είναι 
      άνω φραγμένη, γιατί αν υποθέσουμε ότι είναι, τότε $ \exists M>0 $ ώστε 
      $ a_{n} \leq M, \; \forall n \in \mathbb{N} 
      \Leftrightarrow 2n+5 \leq M, \; \forall n \in \mathbb{N} 
      \Leftrightarrow 2n \leq M-5, \; \forall n \in \mathbb{N}
      \Leftrightarrow n \leq \frac{M-5}{2}, \; \forall n \in \mathbb{N} $, άτοπο, 
      γιατί το $ \mathbb{N} $ δεν είναι άνω φραγμένο.

    \item Η ακολουθία $ a_{1}=2, \; a_{n+1}=2 - \frac{1}{a_{n}}, \forall n \in 
      \mathbb{N}$
      είναι φραγμένη. Προφανώς, $ a_{n}\leq 2, \; \forall n \in \mathbb{N} $.
      Θα δείξουμε ότι $ a_{n}>1, \; \forall n \in \mathbb{N} $. Επειδή η ακολουθία 
      είναι \textbf{αναδρομική}, με επαγωγή, έχουμε:
      \begin{myitemize}[labelindent=1em]
        \item Για $ n=1 $, $ a_{1}=2>1 $, ισχύει. 
        \item Έστω ότι ισχύει για $n$, δηλ. $\inlineequation[eq:
          anadepag1]{a_{n}>1}$.
        \item Θ.δ.ο.\ ισχύει και για $ n+1 $. Πράγματι, από τη 
          σχέση~\eqref{eq: anadepag1}, έχουμε
          \[
            a_{n}>1 \Rightarrow \frac{1}{a_{n}} 
            < 1 \Rightarrow - \frac{1}{a_{n}} > 
            -1 \Rightarrow 2 - \frac{1}{a_{n}} 
            > 2-1 \Rightarrow a_{n+1} > 1.
          \] 
      \end{myitemize}

    \item Να δείξετε ότι η ακολουθία, $ a_{n}=a^{n} $ με $ a>1 $ δεν είναι άνω
      φραγμένη. 
      \begin{proof}(Με άτοπο)

        Έστω ότι η  $ a_{n} $ είναι άνω φραγμένη. Τότε $ \exists M \in \mathbb{R} $ 
        ώστε $ \inlineequation[eq:anfragm]{a^{n} \leq M, \; 
        \forall n \in \mathbb{N}} $.  Επειδή $ a>1 $, έχουμε 
        $ a-1>0 $. Θέτουμε $ \theta =a-1 \Rightarrow a = 1+ \theta $, άρα 
        \begin{equation}\label{eq:antheta}
          a^{n} = (1+ \theta )^{n} \geq 1+ n \theta > n \theta , \; 
          \forall n \in \mathbb{N}  
        \end{equation} 
        Από τις σχέσεις~\ref{eq:anfragm} και~\ref{eq:antheta}, έχουμε
        \[
          n \theta < M, \; \forall n \in \mathbb{N} \Leftrightarrow 
          n < \frac{M}{\theta}, \; \forall n \in \mathbb{N}
        \] 
        άτοπο, γιατί το $ \mathbb{N} $ δεν είναι άνω φραγμένο.
      \end{proof}

    \item Να δείξετε ότι η ακολουθία $ a_{n} = \frac{n^{2}+1}{3n+ \sin^{3}{n}} $ 
      δεν είναι άνω φραγμένη. 
      \begin{proof}
      \item {}
        Έστω ότι η $ a_{n} = \frac{n^{2}+1}{3n+ \sin^{3}{n}} $ είναι άνω 
        φραγμένη. Τότε επειδή είναι και κάτω φραγμένη, από το 0, έχουμε ότι
        $ \exists M>0 \; : \; \abs{a_{n}} \leq M, \; \forall n \in 
        \mathbb{N} $, δηλαδή
        \begin{align*}
          \abs{\frac{n^{2}+1}{3n + \sin^{3}{n}}} \leq M \Leftrightarrow 
          \frac{n^{2}+1}{\abs{3n + \sin^{3}{n}}} \leq M \Leftrightarrow 
          n^{2}+1
               &\leq M \cdot \abs{3n + \sin^{3}{n}} \\ 
               &\leq 3nM + M \cdot \abs{\sin{n}} ^{3} \\
               &\leq 3n M +M, \; \forall n \in \mathbb{N}
        \end{align*} 
        Δηλαδή, 
        \[
          n^{2}-3nM \leq M-1 \Rightarrow n^{2}-3nM < M, \; \forall n \in 
          \mathbb{N}
        \] 
        και συμπληρώνοντας το τετράγωνο έχουμε
        \[
          \left(n - \frac{3}{2} M\right)^{2} < M + \frac{9}{4} M^{2}
          \Rightarrow \abs{n - \frac{3}{2} M} < 
          \sqrt{M + \frac{9}{4} M^{2}}, \; \forall n \in \mathbb{N}
        \]
        οπότε
        \[
          - \sqrt{M + \frac{9}{4} M^{2}}< \underbrace{n- \frac{3}{2} M 
          < \sqrt{M + \frac{9}{4} M^{2}}}, \; \forall n \in \mathbb{N} 
          \Rightarrow n < \frac{3}{2} M + \sqrt{M + \frac{9}{4} M^{2}}, 
          \; \forall n \in \mathbb{N} 
        \] 
        άτοπο, γιατί $ \mathbb{N} $ όχι άνω φραγμένο.
      \end{proof}
  \end{enumerate}
\end{examples}

\begin{prop}
  Η ακολουθία $ a_{n} = \left(1+ \frac{1}{n} \right)^{n}, \; \forall n \in \mathbb{N} $ 
  είναι φραγμένη.
\end{prop}
\begin{proof}
\item {}
  Έχουμε ότι $ a_{n} = \left(1+ \frac{1}{n} \right)^{n} \overset{\text{Bern.}}{>} 
  \left(1+n \frac{1}{n}\right) = 2, \; \forall n \in
  \mathbb{N} $, επομένως είναι κάτω φραγμένη.

  \begin{align*}
    a_{n} = \left(1+ \frac{1}{n}\right)^{n} 
    &= 1 + \binom{n}{1} \frac{1}{n} + \binom{n}{2} \frac{1}{n^{2}} + \cdots + 
    \binom{n}{n} \frac{1}{n^{n}} \\
    &= 1+ \frac{n}{1!} \frac{1}{n} + \frac{n(n-1)}{2!} \frac{1}{n^{2}} + \cdots + 
    \frac{n(n-1)\cdots 2\cdot 1}{n!} \frac{1}{n^{n}} \\
    &=1+ \frac{1}{1!} + \frac{1}{2!} \frac{n-1}{n} + \cdots + 
    \frac{1}{n!} \frac{(n-1)\cdots 2\cdot 1}{n^{n-1}} \\
    &<1+ \frac{1}{1!} + \frac{1}{2!} + \cdots \frac{1}{n!} = b_{n}
  \end{align*}
  Για την ακολουθία $b_{n}$, έχουμε:
  \begin{align*}
    b_{n} &= 1 + \frac{1}{1} + \frac{1}{1\cdot 2} + \frac{1}{1 \cdot 2 \cdot 3} + 
    \cdots + \frac{1}{1\cdot 2 \cdot 3 \cdots n} \\ 
          &< 1 + \frac{1}{1} + \frac{1}{1\cdot 2} + \frac{1}{1 \cdot 2 \cdot 2} + 
          \cdots + \frac{1}{1\cdot 2 \cdot 2 \cdots 2} \\ 
          &= 1 + \left(1 + \frac{1}{2} + \frac{1}{2^{2}} + \cdots + 
          \frac{1}{2^{n-1}}\right) \\
          &= 1 + \frac{1- (\frac{1}{2})^{n}}{1- \frac{1}{2}} = 1 + 2 \left(1 - 
          \frac{1}{2^{n}} \right) = 3 - \frac{1}{2^{n-1}} < 3, 
          \; \forall n \in \mathbb{N}
  \end{align*}
  Επομένως
  \[
    2 < a_{n} < 3, \; \forall n \in \mathbb{N}  
  \]
\end{proof}

\pagebreak

\section{Μονοτονία Ακολουθιών}

\begin{mybox1}
  \begin{dfn}
    Μια ακολουθία $ (a_{n})_{n \in \mathbb{N}} $ ονομάζεται:
    \begin{enumerate}[i)]
      \twocolumnsides{
      \item \textcolor{Col1}{γνησίως αύξουσα} 
        $ \overset{\text{ορ.}}{\Leftrightarrow} a_{n} < a_{n+1}, 
        \quad \forall n \in \mathbb{N}$
      \item \textcolor{Col1}{γνησίως φθίνουσα} 
        $ \overset{\text{ορ.}}{\Leftrightarrow} a_{n} > a_{n+1}, 
        \quad \forall n \in \mathbb{N}$
        }{
      \item \textcolor{Col1}{αύξουσα} 
        $ \overset{\text{ορ.}}{\Leftrightarrow} a_{n} \leq a_{n+1}, 
        \quad \forall n \in \mathbb{N}  $.
      \item \textcolor{Col1}{φθίνουσα} 
        $ \overset{\text{ορ.}}{\Leftrightarrow} a_{n} \geq a_{n+1}, 
        \quad \forall n \in \mathbb{N}  $.
      }
  \end{enumerate}
\end{dfn}
\end{mybox1}

\begin{rems}
\item {}
  \begin{enumerate}[i)]
    \item $ (a_{n})_{n \in \mathbb{N}} $ γνησίως αύξουσα (γνησίως φθίνουσα) $ 
      \Rightarrow (a_{n})_{n \in \mathbb{N}} $ αύξουσα (φθίνουσα) 
    \item $ (a_{n})_{n \in \mathbb{N}} $ φθίνουσα  $ 
      \Rightarrow (a_{n})_{n \in \mathbb{N}} $ άνω φραγμένη, με 
      α.φ.\ το $ a_{1} $  
    \item $ (a_{n})_{n \in \mathbb{N}} $ αύξουσα  $ 
      \Rightarrow (a_{n})_{n \in \mathbb{N}} $ κάτω φραγμένη, με 
      κ.φ.\ το $ a_{1} $  
  \end{enumerate}
\end{rems}


\begin{rem}\label{dfn:isodmono}
\item {}
  Αν μια ακολουθία $ (a_{n})_{n \in \mathbb{N}} $ \textbf{διατηρεί σταθερό πρόσημο}, 
  τότε έχουμε:

  \twocolumnsidesss{
    Αν $ a_{n}>0, \; \forall n \in \mathbb{N} $, τότε:
    \begin{myitemize}[leftmargin=*]
      \item $ \frac{a_{n+1}}{a_{n}} > 1, \; \forall n \in \mathbb{N} \Rightarrow
        (a_{n})_{n \in \mathbb{N}} $ είναι \textcolor{Col1}{γνησίως αύξουσα}  
      \item $ \frac{a_{n+1}}{a_{n}} < 1, \; \forall n \in \mathbb{N} \Rightarrow 
        (a_{n})_{n \in \mathbb{N}} $ είναι \!\textcolor{Col1}{γνησίως φθίνουσα}  
      \item $ \frac{a_{n+1}}{a_{n}} \geq 1, \; \forall n \in \mathbb{N} \Rightarrow 
        (a_{n})_{n \in \mathbb{N}} $ είναι \textcolor{Col1}{αύξουσα} 
      \item $ \frac{a_{n+1}}{a_{n}} \leq 1, \; \forall n \in \mathbb{N} \Rightarrow 
        (a_{n})_{n \in \mathbb{N}} $ είναι \textcolor{Col1}{φθίνουσα}
    \end{myitemize}
    }{
    Αν $ a_{n}<0, \; \forall n \in \mathbb{N} $, τότε:
    \begin{myitemize}[leftmargin=*]
      \item $ \frac{a_{n+1}}{a_{n}} > 1, \; \forall n \in \mathbb{N} \Rightarrow
        (a_{n})_{n \in \mathbb{N}} $ είναι \!\textcolor{Col1}{γνησίως φθίνουσα}  
      \item $ \frac{a_{n+1}}{a_{n}} < 1, \; \forall n \in \mathbb{N} \Rightarrow
        (a_{n})_{n \in \mathbb{N}} $ είναι \textcolor{Col1}{γνησίως αύξουσα}  
      \item $ \frac{a_{n+1}}{a_{n}} \geq 1, \; \forall n \in \mathbb{N} \Rightarrow 
        (a_{n})_{n \in \mathbb{N}} $ είναι \textcolor{Col1}{φθίνουσα}  
      \item $ \frac{a_{n+1}}{a_{n}} \leq 1, \; \forall n \in \mathbb{N} \Rightarrow 
        (a_{n})_{n \in \mathbb{N}} $ είναι \textcolor{Col1}{αύξουσα}  
    \end{myitemize}
  }
\end{rem}

\section{Μεθοδολογία εύρεσης μονοτονίας μιας ακολουθίας}
\begin{myitemize}
  \item Σχηματίζουμε τη διαφορά $ a_{n+1} - a_n $ και ελέγχουμε το πρόσημό της. 
    Αν $ a_{n+1}-a_{n}>0, \; (<0), \; \forall n \in \mathbb{N} $ τότε 
    $ (a_{n})_{n \in \mathbb{N}}$ γνησίως αύξουσα (γνησίως φθίνουσα). 
    Αν για τουλάχιστον ένα $ n \in \mathbb{N} $, στις παραπάνω ανισότητες, 
    έχω ισότητα, τότε  $ (a_{n})_{n \in \mathbb{N}} $ είναι αύξουσα (φθίνουσα).
  \item Αν οι όροι της ακολουϑίας διατηρούν πρόσημο, $ \; \forall n \in \mathbb{N} $ 
    τότε συγκρίνουμε το πηλίκο δυο διαδοχικών όρων της ακολουθίας με τη μονάδα, 
    και βγάζουμε τα συμπεράσματά μας σύμφωνα με την παρατήρηση~\ref{dfn:isodmono}
  \item Αν η ακολουθία δίνεται με μη-αναδρομικό τύπο, και είναι (αρκετά) σύνθετη, 
    τότε μετατρέπω την ακολουθία στην αντίστοιχη συνάρτηση και μελετάμε τη 
    μονοτονία της αντίστοιχης συνάρτησης, συνήθως με τη βοήθεια της παραγώγου.
  \item Αν η $ (a_{n})_{n \in \mathbb{N}} $ δίνεται με αναδρομικό τύπο 
    $ (a_{n+1}= f(a_{n}), \; \forall n \in \mathbb{N}) $ τότε συνήθως η απόδειξή 
    της γίνεται με Μαθηματική Επαγωγή.
\end{myitemize}

\begin{examples}
\item {}
  \begin{enumerate}[i)]
    \item Η $ a_{n} = 2n-1, \; \forall n \in \mathbb{N} $ είναι γνησίως 
      αύξουσα. Πράγματι,

      \twocolumnsides{%
        \begin{description}
          \item[Α᾽ τρόπος: (κατασκευαστικός)]
            \begin{align*}
              n+1 &\geq n, \; \forall n \in \mathbb{N} \\
              2(n+1) &\geq 2n, \; \forall n \in \mathbb{N} \\
              2(n+1) -1 &\geq 2n-1, \; \forall n \in 
              \mathbb{N} \\
              a_{n+1} &\geq a_{n}, \; \forall n \in \mathbb{N}
            \end{align*}
        \end{description}
        }{%
        \begin{description}
          \item[Β᾽ τρόπος: (με το πρόσημο της διαφοράς)] 
            \begin{align*}
              a_{n+1}-a_{n} &= 2(n+1)-1 - (2n-1) \\
                            &= 2 >0, \; \forall n \in 
                            \mathbb{N} 
            \end{align*}
        \end{description}
      }

    \item Η $ a_{n} = \frac{(n-1)!}{n^{n}}, \; \forall n \in 
      \mathbb{N} $ 
      είναι γνησίως φθίνουσα. Πράγματι, επειδή όλοι οι όροι της ακολουθίας 
      είναι \textbf{θετικοί}, επομένως διατηρεί πρόσημο, έχουμε:
      \[
        \frac{a_{n+1}}{a_n} =
        \frac{\frac{(n+1-1)!}{(n+1)^{n+1}}}{\frac{(n-1)!}{n^{n}}} 
        = \frac{n^{n}}{(n+1)^{n+1}}\cdot \frac{n!}{(n-1)!} 
        = \frac{n^{n}}{(n+1)^{n+1}}\cdot \frac{1\cdot 2\cdots (n-1)\cdot n}{1 \cdot 2
        \cdot (n-1)}  
        = \left(\frac{n}{n+1} \right)^{n+1} < 1, \; \forall n \in \mathbb{N} 
      \] 

    \item Η $ a_{n}= \frac{4^{n}}{n^{2}}, \; \forall n \in \mathbb{N} $ 
      είναι αύξουσα. Πράγματι, επειδή όλοι οι όροι της ακολουθίας 
      είναι \textbf{θετικοί}, επομένως διατηρεί πρόσημο, έχουμε:
      \[
        \frac{a_{n+1}}{a_{n}} 
        = \frac{\frac{4^{n+1}}{(n+1)^{2}}}{\frac{4^{n}}{n^{2}}} 
        = \frac{4^{n+1}}{4^{n}}\cdot \frac{n^{2}}{(n+1)^{2}}  
        = \frac{4\cdot \cancel{4^{n}}}{\cancel{4^{n}}} \cdot \frac{n^{2}}{(n+1)^{2}}
        = \left( \frac{2n}{n+1} \right)^{2} \geq 1, 
        \; \forall n \in \mathbb{N} 
      \]
      Πράγματι, η ακολουθία δεν είναι γνησίως αύξουσα, γιατί $ a_{1}= a_{2}=4$.

    \item Η $ a_{n+1}=2 - \frac{1}{a_{n}}, \; \forall n \in \mathbb{N}
      $ με $ a_{1} = 2 $ είναι γνησίως φθίνουσα. Πράγματι, επειδή η ακολουθία είναι 
      \textbf{αναδρομική}, με μαθηματική επαγωγή, έχουμε:
      \begin{myitemize}[labelindent=1em]
        \item Για $ n=1 $, έχω: $ a_{1}= 2 >
          \frac{3}{2} = 2 - \frac{1}{2} = a_{2}$, ισχύει.
        \item Έστω ότι ισχύει για $n$, δηλ.
          $\inlineequation[eq:epag]{a_{n+1}<a_{n}}$.
        \item Θ.δ.ο.\ ισχύει για $ n+1 $. Πράγματι, από τη σχέση~\eqref{eq:epag} 
          έχουμε:
          \begin{align*}
            a_{n+1} < a_{n} \Leftrightarrow \frac{1}{a_{n+1}} > \frac{1}{a_{n}}
            \Leftrightarrow - \frac{1}{a_{n+1}} < - \frac{1}{a_{n}} \Leftrightarrow 2-
            \frac{1}{a_{n+1}} < 2 - \frac{1}{a_{n}} \Leftrightarrow a_{(n+1)+1} 
            < a_{n+1}
          \end{align*} 
      \end{myitemize}

    \item Να δείξετε ότι ακολουθία $ a_{n} = (-1)^{n} \frac{1}{n^{2}}, 
      \; \forall n \in \mathbb{N} $ 
      δεν είναι μονότονη.

      \begin{proof}
        Θα δείξουμε ότι η ακολουθία δεν διατηρεί πρόσημο. Πράγματι, 
        $ \forall n \in \mathbb{N} $
        \begin{align*}
          a_{n+1}- a_{n} = \frac{(-1)^{n+1}}{(n+1)^{2}} - 
          \frac{(-1)^{n}} {n^{2}} = \frac{(-1)^{n+1}}{(n+1)^{2}} + 
          \frac{(-1)^{n+1}}{n^{2}} 
          = (-1)^{n+1}\Biggl[\underbrace{\frac{1}{(n+1)^{2}} + 
              \frac{1}{n^{2}}}_{b_{n} > 0, \; \forall n \in 
          \mathbb{N}}\Biggr] 
          = \begin{cases}
            b_{n}, & n \; \text{περιττός} \\
            -b_{n}, & n \; \text{άρτιος} 
          \end{cases}
        \end{align*} 
      \end{proof}
  \end{enumerate}
\end{examples}

% \begin{mybox3}
% \begin{prop}
% Η ακολουθία $ \left(1+ \frac{1}{n}\right)^{n} $ είναι γνησίως αύξουσα.
% \end{prop}
% \end{mybox3}
% \begin{proof}
% \item {}
%   Έστω $ n_{0} \in \mathbb{N} $. Τότε 
%   \begin{align*}
%     \left(1+ \frac{1}{n_{0}+1} \right)^{n_{0}+1} > 
%     \left(1+ \frac{1}{n_{0}} \right)^{n_{0}} 
%         &\Leftrightarrow \left(1+ \frac{1}{n_{0}+1} \right)^{n_{0}} 
%         \cdot \left(1 + \frac{1}{n_{0} +1} \right) > \left(1+ \frac{1}{n_{0}+1} 
%         \right) \\
%         & \Leftrightarrow \frac{(n_{0}+2)^{n_{0}}}{(n_{0}+1)^{n_{0}}} \cdot 
%         \frac{n_{0}+2}{n_{0}+1} > \frac{(n_{0}+1)^{n_{0}}}{n_{0}^{n_{0}}} \\
%         & \Leftrightarrow \frac{n_{0}^{n_{0}}(n_{0}+2)^{n_{0}}}{(n_{0}+1)^{2n0}} > 
%         \frac{n_{0}+1}{n_{0}+2} \\
%         & \Leftrightarrow \left(\frac{n_{0}(n_{0}+2)}{(n_{0}+1)^{2}}\right)^{n_{0}} > 
%         \frac{n_{0}+2-2+1}{n_{0}+2} \\
%         & \Leftrightarrow \left(\frac{n_{0}^{2}+2 n_{0}+1-1}{(n_{0}+1)^{2}}\right)
%         ^{n_{0}} > 1-\frac{1}{n_{0}+2} \\
%         & \Leftrightarrow \left(\frac{(n_{0}+1)^{2}-1}{(n_{0}+1)^{2}} \right)^{n_{0}}
%         > 1 - \frac{1}{n_{0}+2} \\
%         & \Leftrightarrow \left(1 - \frac{1}{(n_{0}+1)^{2}} \right)^{n_{0} } > 1 - 
%         \frac{1}{n_{0}+2} \\
%   \end{align*} 
%   Άρα αρκεί να δείξουμε αυτή την ανισότητα. Πράγματι,
%   για $ a = - \frac{1}{(n_{0}+1)^{2}} > -1 $ από ανισότητα Bernoulli: 
%   \begin{align*}
%     \left(1- \frac{1}{(n_{0}+1)^{2}}\right)^{n_{0}} > 1 - n_{0}\cdot 
%     \frac{1}{(n_{0}+1)^{2}} > 1 - \frac{1}{n_{0}+2} 
%   \end{align*}
%   όπου χρησιμοποιήσαμε το γεγονός ότι $ \frac{n_{0}}{(n_{0}+1)^{2}} < 
%   \frac{1}{n_{0}+2}  $, το οποίο ισχύει, γιατί 
%   \[
%     \frac{n_{0}}{(n_{0}+1)^{2}} < \frac{1}{n_{0}+2} 
%     \Leftrightarrow n_{0}(n_{0}+2) < (n_{0}+1)^{2} 
%     \Leftrightarrow n_{0}^{2}+2 n_{0} < n_{0}^{2} + 2 n_{0}+1 
%     \Leftrightarrow 0 < 1
%   \]
% \end{proof}

\begin{prop}
  Η ακολουθία $ a_{n} = \left(1+ \frac{1}{n} \right)^{n}, \; \forall n \in \mathbb{N} $ 
  είναι γνησίως αύξουσα.
\end{prop}
\begin{proof}
  Έχουμε ότι $ a_{n} > 0, \; \forall n \in \mathbb{N} $.  
  \begin{align*}
    \frac{a_{n+1}}{a_{n}} &= \frac{\left(1+ \frac{1}{n+1} \right)^{n+1}}{\left(1+ \frac{1}{n}\right)^{n}} 
    = \left(1+ \frac{1}{n}\right)\left(\frac{1+ \frac{1}{n+1}}{1+
    \frac{1}{n}}\right)^{n+1} 
    = \left(1+ \frac{1}{n}\right) \left(\frac{n(n+2)}{(n+1)^{2}}\right)^{n+1} \\
                          &= \left(1+
                          \frac{1}{n}\right)\left(\frac{n^{2}+2n\color{Col1}{+1-1}}{(n+1)^{2}}\right)^{n+1}
                          = \left(1+ \frac{1}{n} \right) \left(1- \frac{1}{(n+1)^{2}}
                          \right)^{n+1 } \\
                          & \smash{\overset{\text{Bern.}}{>}} \left(1+ \frac{1}{n}\right) \left(1-(n+1)
                          \frac{1}{(n+1)^{2}}\right) = \left(1+ \frac{1}{n}\right) 
                          \left(1- \frac{1}{n+1}\right) = \frac{n+1}{n} \cdot
                          \frac{n}{n+1} = 1, \; \forall n \in \mathbb{N}
  \end{align*}
\end{proof}



\end{document}
